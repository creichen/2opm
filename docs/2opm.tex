\documentclass{article}
\usepackage{color}
\usepackage{listings}
\usepackage{2opm}
\definecolor{dblue}{rgb}{0,0,0.5}

\newcommand{\textdbf}[1]{\texttt{\textcolor{dblue}{\textbf{#1}}}}

\lstset{ %
  language=[2opm]Assembler,       % the language of the code
  basicstyle=\small\tt,       % the size of the fonts that are used for the code
%  numbers=left,                   % where to put the line-numbers
  numberstyle=\small\color{grey},  % the style that is used for the line-numbers
  stepnumber=1,                   % the step between two line-numbers. If it's 1, each line 
                                  % will be numbered
  numbersep=5pt,                  % how far the line-numbers are from the code
  backgroundcolor=\color{white},  % choose the background color. You must add \usepackage{color}
  showspaces=false,               % show spaces adding particular underscores
  showstringspaces=false,         % underline spaces within strings
  showtabs=false,                 % show tabs within strings adding particular underscores
%  frame=single,                   % adds a frame around the code
  rulecolor=\color{black},        % if not set, the frame-color may be changed on line-breaks within not-black text (e.g. commens (green here))
  tabsize=4,                      % sets default tabsize to 2 spaces
  captionpos=b,                   % sets the caption-position to bottom
  breaklines=true,                % sets automatic line breaking
  breakatwhitespace=false,        % sets if automatic breaks should only happen at whitespace
  title=\lstname,                 % show the filename of files included with \lstinputlisting;
                                  % also try caption instead of title
  keywordstyle=\textdbf,           % keyword style
  commentstyle=\color{dgreen},       % comment style
  stringstyle=\color{mauve},         % string literal style
  escapeinside={\%*}{*)},            % if you want to add a comment within your code
  morekeywords={*,eingabe,ausgabe,...},               % if you want to add more keywords to the set
  belowskip=-1em
}

\newcommand{\insn}[1]{\texttt{\textbf{\textcolor{dblue}{#1}}}}
\title{2OPM assembly instruction list\\ \normalsize v0.1.1}

\begin{document}
\maketitle

2OPM is an assembly language intended for immediate execution on
x86-64 CPUs. It is not interpreted but rather translated directly into
native machine code.

\section{Registers}
2OPM Assembly uses the following general-purpose registers:


  \begin{tabular}{|p{6cm}|l|}
    \hline
    \textbf{Name} & \textbf{Purpose} \\
    \hline
    \hline
    \texttt{\$v0} & Return \textbf{V}alue\\
    \texttt{\$a0}, \texttt{\$a1}, \texttt{\$a2}, \texttt{\$a3}, \texttt{\$a4}, \texttt{\$a5} & \textbf{A}rguments\\
    \texttt{\$s0}, \texttt{\$s1}, \texttt{\$s2}, \texttt{\$s3} & \textbf{S}aved registers \\
    \texttt{\$t0}, \texttt{\$t1} & \textbf{T}emporary registers \\
    \texttt{\$sp} & \textbf{S}tack \textbf{P}ointer\\
    \texttt{\$fp} & \textbf{F}rame \textbf{P}ointer\\
    \texttt{\$gp} & \textbf{G}lobal \textbf{P}ointer\\
    \hline
  \end{tabular}


Furthermore, it uses the special-purpose \texttt{\$pc} (\emph{program counter} register to indicate
the address of the next instruction to execute.  Regular instructions cannot access this register directly,
though jump and branch operations can modify it.

When loading an assembly program, the loader ensures that the following registers are set to reasonable
addresses before program start:
\begin{itemize}
  \item \texttt{\$sp} is a viable stack address, and initially \texttt{\$sp} modulo 16 is 0 (as after a subroutine call).
  \item \texttt{\$gp} points to a special static memory region for the program.
\end{itemize}
Furthermore, when entering an assembly program, the loader places a viable return address on the stack,
so that assembly programs can terminate with \texttt{jreturn}.

\section{Memory}
The loader provides two special memory segments to the loaded assembly program:
\begin{itemize}
\item A custom code region (used implicitly by the program counter)
\item A custom static memory region (referenced by \texttt{\$gp}).
\end{itemize}
The assembly program re-uses the loader's stack.  At present, direct heap access via assembly is not intended.

\section{Instructions}

Assembly programs consist of sequences of assembly instructions.  The instructions are listed below, along with brief explanations.
Each instruction may take a number of arguments.  We distinguish between the following kinds of arguments:
\begin{itemize}
  \item \texttt{addr}: A memory address (usually passed in by a label)
  \item \texttt{u8}: An 8-bit unsigned number
  \item \texttt{u32}: A 32-bit unsigned number
  \item \texttt{s32}: A 32-bit signed number
  \item \texttt{u64}: A 64-bit unsigned number
  \item \texttt{\$r0}, \texttt{\$r1}, \texttt{\$r2}: Any general-purpose register
\end{itemize}

\pagenumbering{gobble}
\input{asm-ops}
\newpage
\pagenumbering{arabic}

\section{Calling conventions}

2OPM follows the x86-64/Linux ABI (Application Binary Interface), translated to 2OPM's register names.

\subsection{Preparations before subroutine call}
  \begin{itemize}
    \item First six parameters in \texttt{\$a0}\ldots\texttt{\$a5}
    \item Additional parameters in memory:
      \begin{itemize}
        \item Argument 6 (7th): in memory at \texttt{\$sp}
        \item Argument 7 (8th): in memory at $\texttt{\$sp} + 8$\\
          \ldots
      \end{itemize}
    \item $\texttt{\$sp} + 8$ is \emph{$128$ bit aligned} (divisible by $16$)
  \end{itemize}

\subsection{When entering a subroutine}
    \begin{itemize}
    \item $\texttt{\$sp}$ is \emph{$128$ bit aligned}
    \item Memory at \texttt{\$sp} contains return address
    \end{itemize}

    The \insn{jal} instruction ensures these properties implicitly if the assembly
    program makes the correct assumptions prior to the subroutine call.

\subsection{During subroutine execution}
    \begin{itemize}
    \item $\texttt{\$fp} + 8$ is \emph{$128$ bit aligned}
    \item Function has \emph{stack frame}:
      \begin{itemize}
        \item Argument 7 (8th): at \texttt{\$fp} + 24 (etc.)
        \item Argument 6 (7th): at \texttt{\$fp} + 16
        \item Return address at \texttt{\$fp} + 8
        \item Caller's \texttt{\$fp} at \texttt{\$fp}
        \item Local variables: start at \texttt{\$fp} - 8
      \end{itemize}
    \end{itemize}

    The ABI permits not storing \texttt{\$fp} as an optional
    optimisation, where feasible.
    In that case, local variables start directly
    at the memory address indicated by \texttt{\$fp}.

\subsection{After return from a subroutine}
  \begin{itemize}
    \item The following \emph{callee-saved registers} have the same contents as before the call:
      \begin{itemize}
        \item \texttt{\$sp}
        \item \texttt{\$fp}
        \item \texttt{\$gp}
        \item \texttt{\$s0}--\texttt{\$s3}
      \end{itemize}
    \item All other registers \emph{may be modified}
    \item Register \texttt{\$v0} contains the return value, if any
  \end{itemize}

\section{Command-line Assembler Tool}

The 2OPM command line assembler (\texttt{asm}) can load, link, and run assembly files
(using the suffix \texttt{.s}, by convention).  It also includes a debugger that can step
through code, print out registers, stack contents, and static memory, and execute until it hits a breakpoint\footnote{
Only one breakpoint is supported at present.}

\subsection{Installing the Assembler}
Download the assembler from the specified location.  If you unpack it
on a UNIX command line and run `\texttt{make}', it should compile
and link a program `\texttt{asm}'.  This program is a stand-alone executable
and can be run from any location.

\subsection{Using the Assembler}
To start the assembler, write a small assembly program, store it in the file \texttt{myprogram.s} in the same
directory that contains your \texttt{asm} executable, and run
\[
\texttt{./asm myprogram.s}
\]

on the command shell in that directory (see below for some sample programs).

\subsubsection{Using the Debugger}
To activate the debugger, start the assembler with the command line option \texttt{-d}.
The debugger has a built-in help facility that can be activated by writing \texttt{help} as
soon as the debugger command prompt appears.

\subsection{Assembler Programs}

The assembler takes four kinds of input:
\begin{itemize}
  \item \emph{assembly instructions},
  \item \emph{labels},
  \item \emph{directives}, which control the meaning of subsequent input, and
  \item \emph{data}.
\end{itemize}

A functional program must provide at least the first three; providing
data is optional.

As an example, consider this program:

\begin{lstlisting}
.text
main:
        push $t0        ; align stack for subroutine call
        li   $a0, 42
        jal  print_int  ; call built-in function to print
        pop  $t0
        jreturn
\end{lstlisting}
(Note the comment syntax.)

This program consists of five assembly instructions, one of which
calls a built-in function (see below).  The first two lines, however,
are not assembly instructions.  Here, \texttt{.text} indicates that
the following output should go into the text segment.  The assembler
will permit assembly instructions if and only if the text segment has
been selected.  The label \texttt{main} marks the main entry point.
Each executable assembly program MUST define a \texttt{main} entry
point.  Any further labels are optional.

\subsection{Directives}

2OPM supports five directives.  The two most important ones are \texttt{.text} and \texttt{.data}.

\paragraph{\texttt{.text}} indicates that any following information goes into the text segment, 
i.e., is intended for execution.  The following information must be assembly instructions
and may include label definitions and label references (for suitable instructions).

\paragraph{\texttt{.data}} indicates  that any following information is pure data.  No 
assembly instructions are permitted (this is for simplicity; in principle, the
computer could represent assembly instructions in static memory).  The data section 
permits label definitions, and
freely
mixes all forms of data; however, introducing data requires selecting a \emph{data mode}.

\subsection{Data modes}

The following data modes are permitted:

\paragraph{\texttt{.byte}} allows inserting single bytes, separated by commas.

\paragraph{\texttt{.word}} allows inserting 64-bit words, separated by commas.  This section
also permits label references: the labels' memory address are included verbatim.

\paragraph{\texttt{.asciiz}} allows ASCII character strings.  All strings are automatically
zero-terminated.

As an example for using the data segment, consider the following:

\begin{lstlisting}
.text
main:
        push $fp      ; align stack
        move $fp, $sp

        la   $a0, hello
        jal  print_string  ; print out
        ld   $a0, number($gp)
        jal  print_int
        la   $t0, more_numbers
        ld   $v0, 0($t0)
        ld   $a0, 8($t0)
        add  $a0, $v0
        jal  print_int     ; print sum

        pop  $fp
        jreturn
.data
hello:
.asciiz "Hello, World!"
.word
number:
	23
more_numbers:
	3,4
\end{lstlisting}

\subsection{Labels}

Labels are defined by writing the label's name, followed by a colon, as in \texttt{label:}.
References to labels are written by omitting the colon.  Each label may be defined only once,
but may be referenced arbitrarily many times.

Label references are permitted in \texttt{.data} sections in \texttt{.word} mode, and in
assembly instructions such as \texttt{\textcolor{dblue}{j}} or \texttt{\textcolor{dblue}{blt}}.  The
assembler automatically figures out whether the references are relative or absolute and relocates suitably.

To load a memory address of any label directly, the assembler provides a pseudo-instruction:

\[
\texttt{\textcolor{dblue}{la} \texttt{\$r}, \texttt{\textit{addr}}}
\]

This instruction loads the absolute memory address of the specified label into register \texttt{\$r},
no matter whether the address is in text or (static) data memory.


\subsubsection{Built-in Operations}
The 2OPM runtime comes with a small number of built-in subroutines to facilitate
input and output.  Each of them can be called using
\texttt{\textcolor{dblue}{jal}}:

\begin{tabular}{ll}
\texttt{print\_int} & Print parameter as signed integer \\
\texttt{print\_string} & Print parameter as null-terminated ascii string \\
\texttt{read\_int} & Read and return a single 64 bit integer \\
\texttt{exit} & Stop the program \\
\end{tabular}

These functions use system calls (cf. the
\texttt{\textcolor{dblue}{syscall}} operation) to achieve special
effects that require interaction with the operating system; they are
abstracted for convenience.


For example, the following program will read two numbers, add them,
and print the resultant output:

\begin{lstlisting}
.text
main:
        push $fp
        move $fp, $sp ; align stack
        jal  read_int
        move $s0, $v0
        jal  read_int
        move $a0, $s0
        add  $a0, $v0
        jal  print_int
        pop  $fp
        jreturn
\end{lstlisting}


\section{Implementation Notes}

Most 2OPM instructions correspond directly to x86-64 instructions.
However, some of the instruction choices are not optimal: for example,
2OPM always uses 64 bit load operations, even if the number loaded can
be represented in 32 or fewer bits.  Some other operations are
emulated: x86-64 only permits bit shifting by register \texttt{cl}
(\texttt{\$a2}[7:0]), so the 2OPM implementation of this instruction
introduces additional register-swap operations, if needed.  This may
result in less-than-optimal performance.

\end{document}
